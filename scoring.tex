\documentclass{article}

\usepackage[utf8x]{inputenc}
\usepackage[T1]{fontenc}

\usepackage[box,completemulti]{automultiplechoice}

\usepackage{listings}

\lstset{%
  basicstyle=\small\ttfamily,
  breaklines=true,
  columns=fullflexible,
  frame=single,
  frameround=tttt,
  showstringspaces=false
}



\begin{document}

\element{qqs}{
\begin{question}{rr}
  \\~\\
  What would be the result of : isNaN(true)
  \begin{choices}
    \wrongchoice{true}\scoring{0}
    \correctchoice{false}\scoring{1}
    \wrongchoice{NaN}\scoring{0}
    \wrongchoice{undefined}\scoring{0}
  \end{choices}
\end{question}
}

\element{qqs}{
\begin{question}{ss}
	\\~\\
  What would be the result of : Boolean("false") 
  \begin{choices}
    \correctchoice{true}\scoring{1}
    \wrongchoice{false}\scoring{0}
    \wrongchoice{boolean}\scoring{0}
    \wrongchoice{undefined}\scoring{0}
  \end{choices}
\end{question}
}

\element{qqs}{
\begin{question}{xx}
	\\~\\
  The output of the comparison \{\} == \{\} will be :
  \begin{choices}
    \correctchoice{false}\scoring{1}
    \wrongchoice{true}\scoring{0}
    \wrongchoice{Illegal comparison}\scoring{0}
    \wrongchoice{undefined is not a function}\scoring{0}
  \end{choices}
\end{question}
}

\element{qqs}{
\begin{question}{xsxx}
	\\~\\
  > var undefined = 5; \\
  > console.log(undefined); 
  \\~\\
  After executing these instructions, The printed result will be : 
  \begin{choices}
    \wrongchoice{5}\scoring{0}
    \wrongchoice{random output}\scoring{0}
    \correctchoice{undefined}\scoring{1}
    \wrongchoice{Uncaught ReferenceError: undefined is not defined}\scoring{0}
  \end{choices}
\end{question}
}

\element{qqs}{
\begin{question}{xxdx}
  \\~\\
  What will be the result of : typeof NaN
  \\
  \begin{choices}
    \wrongchoice{undefined}\scoring{0}
    \wrongchoice{NaN}\scoring{0}
    \correctchoice{number}\scoring{1}
    \wrongchoice{object}\scoring{0}
  \end{choices}
\end{question}
}



\element{qqs}{
\begin{question}{xxfx}
  \\~\\
  After executing these instructions, what will be the output : 
  \\
  \\>var myArray = Array.apply(null, $\left\{length: 5\right\}$).map(Number.call, Number);
  \\>myArray.push(7);
  \\>myArray.splice(2,1);
  \\>myArray.shift();
  \\>myArray.pop();
  \\>console.log(myArray);
  \begin{choices}
    \wrongchoice{[ 1, 4, 6 ]}\scoring{0}
    \wrongchoice{[ 1, 2, 3, 4, 5 ]}\scoring{0}
    \correctchoice{[ 1, 3, 4 ]}\scoring{1}
    \wrongchoice{[ 1, 2, 3 ]}\scoring{0}
  \end{choices}
\end{question}
}

\element{qqs}{
\begin{question}{let}
  \\~\\
  What's the main difference between 'var' and 'let' 
  \begin{choices}
    \wrongchoice{'let' is the new 'var'}\scoring{0}
    \wrongchoice{scoping and visibility of variables}\scoring{1}
    \correctchoice{They are identical everywhere}\scoring{0}
    \wrongchoice{'let' is used for static declarations while 'var' is not}\scoring{0}
  \end{choices}
\end{question}
}

\element{qqs}{
\begin{question}{stamp}
  \\~\\
  What would be the output of this instruction : new Date('1457004845000')
  \begin{choices}
    \wrongchoice{No output, because 1457004845000 is a Timestamp in milliseconds}\scoring{0}
    \wrongchoice{Thu Jan 01 1970 01:00:00 GMT+0100 (CET)}\scoring{0}
    \correctchoice{Invalid Date}\scoring{1}
    \wrongchoice{Thu Mar 03 2016 12:34:05 GMT+0100 (CET)}\scoring{0}
  \end{choices}
\end{question}
}


\element{qqs}{
\begin{question}{pow}
  \\~\\
  In javascript, the range of integers that can be safely handled is : 
  \begin{choices}
    \correctchoice{[$-2^{53}$,$2^{53}$]}\scoring{0}
    \wrongchoice{[$-2^{32}$,$2^{32}$]}\scoring{0}
    \wrongchoice{[$-2^{64}$,$2^{64}$]}\scoring{1}
    \wrongchoice{[$-2^{33}$,$2^{33}$]}\scoring{0}
  \end{choices}
\end{question}
}

\element{qqs}{
\begin{question}{sum}
  \\~\\
  What will be the sum of 0.1 and 0.2 : 
  \begin{choices}
    \wrongchoice{0.3}\scoring{0}
    \wrongchoice{0.30000000000000003}\scoring{0}
    \correctchoice{0.30000000000000004}\scoring{1}
    \wrongchoice{0.299999999999999999}\scoring{0}
  \end{choices}
\end{question}
}

\element{qqs}{
\begin{question}{ang}
   
  \lstinputlisting[%
    language=HTML,
    stringstyle=\color{blue}
  ]{templates/angular.html}
  
  After executing this code, you will get :
  
  \begin{choices}
    \wrongchoice{A rendered HTML : message}\scoring{0}
    \wrongchoice{A rendered HTML : a blank page}\scoring{0}
    \correctchoice{Error in the Console : Infinite \$digest Loop}\scoring{1}
    \wrongchoice{Error in the Console : Syntax error}\scoring{0}
  \end{choices}
\end{question}
}

\element{qqs}{
\begin{question}{sort}
   
  What will be the output of these instructions :
  > let arr = [9,1,2,18]; arr.sort(); console.log(arr);
  
  \begin{choices}
    \wrongchoice{[ 18, 9, 2, 1 ]}\scoring{0}
    \wrongchoice{[ 1, 2, 9, 18 ]}\scoring{0}
    \correctchoice{[ 1, 18, 2, 9 ]}\scoring{1}
    \wrongchoice{[ 18, 1, 9, 2 ]}\scoring{0}
  \end{choices}
\end{question}
}


\element{qqs}{
\begin{question}{condition}


   The output of this script will be : 
   
  let isValid = 0;

  if (isValid \&\& isValid !== null \&\& isValid !== undefined) \{
  
  \qquad console.log("A");
      
\} else if (!isValid \&\& isValid !== null \&\& isNaN(!isValid) === false) \{

  \qquad console.log("B")
    
\} else if (!isValid \&\& isValid !== null \&\& isValid !== undefined) \{

  \qquad console.log("C")
    
\} else \{

  \qquad console.log("D")
    
\}

  
  
  \begin{choices}
    \wrongchoice{A}\scoring{0}
    \correctchoice{B}\scoring{1}
    \wrongchoice{C}\scoring{0}
    \wrongchoice{D}\scoring{0}
  \end{choices}
\end{question}
}


%%%%%%%%%%%%%%%%%%%%%%%%%%%%%%%%%%%%%%%%%%%%%%%%%%%%%%%%%%%%%%%%%%%%%%

\onecopy{20}{

\noindent{\bf QCM  \hfill Good Luck} 

\vspace*{.5cm}
\begin{minipage}{.4\linewidth}
\centering\large\bf Orion Code \\ \small{Copyrighted content}\end{minipage}
\namefield{\fbox{\begin{minipage}{.6\linewidth}
Full candidate name:

\vspace*{.5cm}\dotfill
\vspace*{10mm}
\end{minipage}}}

\vspace*{5mm}

%%%%%%%%%%%%%%%%%%%%%%%%%%%%%%%%%%%%%%%%%%%%%%%%%%%%%%%%%%%%%%%%%%%%%%

\shufflegroup{qqs}

\insertgroup{qqs}

%%%%%%%%%%%%%%%%%%%%%%%%%%%%%%%%%%%%%%%%%%%%%%%%%%%%%%%%%%%%%%%%%%%%%%

\clearpage

}

\end{document}
